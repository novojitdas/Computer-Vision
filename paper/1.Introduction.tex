
\chapter{Introduction}
Waste management has emerged as a pressing global concern, with implications for environmental sustainability, public health, and economic development. Particularly urban areas are finding it difficult to manage the enormous amount of waste produced by their expanding industry and inhabitants. For waste to have as little negative influence as possible on the environment and public health, effective waste management techniques are essential. 

Dhaka, the capital city of Bangladesh, exemplifies the challenges faced by many metropolitan areas. Every day, Dhaka produces approximately 500 tonnes of waste[1]. This staggering amount poses significant challenges for the city's waste management infrastructure, which is often inadequate to handle such large quantities efficiently and sustainably. On a global scale, the situation is even more alarming. It is estimated that more than 2.01 billion tonnes of municipal solid waste are generated annually worldwide[1]. Approximately one-third of total, or 670 million tonnes, is not managed in a way that is safe for the environment. When waste is not properly managed, it causes serious environmental harm, such as air, water, and soil contamination, and it exacerbates climate change by releasing greenhouse gases into the atmosphere when waste breaks down in landfills.

Computer vision, a field of artificial intelligence, enables machines to interpret and make decisions based on visual data. Through the automation of tedious duties that have conventionally been executed by humans, computer vision streamlines and improves the effectiveness of waste classification and management procedures. The primary challenge in waste management is the accurate and efficient sorting of waste into various categories such as recyclables, compostables, and landfill waste. Manual sorting is labor-intensive, error-prone, and inefficient. The sorting procedure can be considerably enhanced by training computer vision systems to quickly and accurately identify and classify waste items.

Our time is predominately data-driven, and waste management is not an exception. Our computer vision models require constant upgrades and enhancements due to the daily generation of new waste data kinds. Ensuring the accuracy and efficacy of our waste sorting systems requires regular training of deep learning models with a variety of current waste photos. The constant flood of fresh data, however, poses a serious problem. It's difficult to train models on big datasets quickly and effectively; it takes a lot of computing power and advanced data management techniques. Our approach to overcoming this obstacle is the application of a stacked model structure.

\section{Motivation}
The environment is seriously harmed by inefficient waste management techniques. According to [2], improper waste management exacerbates climate change by releasing greenhouse gasses like methane from decomposing organic waste in landfills and polluting the air, water, and soil. The utilization of computer vision technology in waste management has the potential to improve waste sorting accuracy, leading to higher recycling rates and a less environmental impact associated with waste disposal. Sorting systems that operate automatically guarantee that waste is handled properly, reducing negative environmental effects.

The improper disposal of waste poses serious risks to public health. Inadequate waste management can lead to the contamination of water sources, proliferation of disease vectors, and exposure to hazardous substances. The growth of a sustainable economy depends on efficient waste management. When waste is not properly managed, it can lead to unhygienic environments, lower property values, and excessive expenditures for towns. 

our motivation is driven by the need to tackle the pressing issues of environmental sustainability, public health, and economic development associated with waste management.

\section{Challenges}
The integration of computer vision into waste management systems presents several significant challenges. These challenges include the handling of large amounts of data, managing diverse classes of waste, the necessity for frequent training and analysis, computational limitations, limited training time, and economic constraints.

Managing waste with computer vision presents several obstacles, one of which is the enormous amount of data that is produced every day. To keep the system responsive and efficient, managing and storing this data calls for a sizable infrastructure and advanced data management strategies. Waste can vary significantly in composition, shape, size, and material. This diversity poses a challenge for computer vision systems, which must be trained to recognize and accurately sort a wide range of waste items. Frequent training and analysis of computer vision models is necessary due to the dynamic nature of waste streams. Frequent emergence of new waste kinds necessitates system adaptation in order to preserve accuracy and dependability.

\section{Research Domain}
Core of this research domain lies artificial intelligence (AI) and computer vision technologies. Cameras and sensors deployed in waste management facilities provide visual data that AI systems, especially deep learning models, are trained to evaluate. To automatically recognize, categorize, and arrange various waste products, computer vision algorithms process these images. Scholars in this field investigate diverse methodologies for enhancing these algorithms' scalability, accuracy, and real-time processing capabilities.[3]

\section{Research Objective}
The goal of this study is to investigate the complex nature of waste, viewing it as an essential component of ecological systems, interspecies connections, and planetary processes rather than just a byproduct of human activity. The goal of this research is to examine the various aspects of waste, such as how it affects sociocultural dynamics, environmental sustainability, and human health.[3]

The goal of this research is to use the largest waste dataset available to create and apply a classification model that can reliably detect and categorize waste into 24 separate types. In order to promote faster training and enhanced performance in waste categorization, this work suggests and evaluates a stacked model structure.

The study aims to tackle the difficulties involved in handling vast quantities of heterogeneous waste data by utilizing sophisticated computer vision and deep learning methodologies. Improving waste sorting systems' accuracy and efficiency is the aim, which will lead to more sustainable and efficient waste management techniques.

\section{Research Outcome}
The research successfully developed a stacked model architecture to address the challenge of prolonged training times associated with large waste datasets. This innovative approach significantly enhanced both the accuracy and efficiency of the classification process. Specifically, the stacked model resulted in substantially faster training times compared to traditional models, allowing for more rapid processing of large volumes of waste data. Additionally, the models demonstrated high accuracy in classifying waste into 24 distinct categories, ensuring reliable and precise waste sorting. The improved training efficiency and accuracy make the model highly scalable and practical for real-world waste management applications. These outcomes indicate that the stacked model architecture is an effective solution for optimizing waste classification systems, contributing to more efficient and sustainable waste management practices.

\section{Background Study}
urban areas continue to expand and industrial activities increase, the volume of waste generated presents substantial challenges for effective and sustainable management practices. The integration of advanced technologies, particularly in the field of computer vision, offers promising solutions to enhance the efficiency and accuracy of waste sorting and classification. This background study reviews the current state of research in this area, highlighting key advancements and identifying gaps that this thesis aims to address.

\subsection{Technological Advancements in Waste Classification}

Recent studies have explored various machine learning and computer vision techniques to automate waste classification, achieving notable improvements in both speed and accuracy. One significant contribution in this field is the development of LitterNets, an ensemble of Convolutional Neural Network (CNN) and Extreme Learning Machine (ELM) models designed for automatic waste classification. LitterNets was trained on a single epoch, resulting in a substantial reduction in training time without sacrificing performance. Remarkably, the model achieved comparable accuracy to existing state-of-the-art models despite not employing data augmentation techniques. This approach leverages pre-trained CNNs as feature extractors and uses ELM for the classification layer, combining multiple trained CNN-ELM models to enhance accuracy. Future research directions suggested by this study include extending the framework to classify other objects and exploring different methodologies for more effective model selection in ensemble architectures [4].

\subsection{Performance Comparison of Waste Detection Models}

Another study focused on reducing trash inputs to local waterways and the ocean by comparing the trash detection performance of three different models. The best-performing model, Mask R-CNN, achieved impressive results with 91\% recall, 83\% precision, and 77\% accuracy using data collected along 84 road segments in two California cities. This study underscores the potential of advanced detection models in accurately identifying and classifying waste in diverse environmental settings, thereby contributing to more efficient waste management practices and environmental conservation efforts [5].

\subsection{Few-Shot Learning for Waste Detection}
A further advancement in this domain is the application of few-shot learning techniques to waste detection. One study proposed a few-shot waste detection framework based on Faster R-CNN, achieving a mean average precision of 31.16\% over 12 waste categories with only 30 instances per category. This framework includes a waste proposal module and a waste classification module, demonstrating the feasibility of effective waste detection even with limited training samples. The ability to achieve reasonable accuracy with few samples is particularly beneficial for waste categories where large labeled datasets are not available, highlighting the potential of few-shot learning in addressing data scarcity in waste classification tasks [6].

\section{Types of tools}

\subsection{Tensorflow}
TensorFlow [7], an open-source machine learning framework by Google Brain, is pivotal in developing machine learning models due to its flexibility, scalability, and comprehensive ecosystem. It supports various algorithms and offers robust tools for building and training models, making it ideal for computer vision applications. In waste classification, TensorFlow's high-level APIs like Keras simplify creating and training complex neural networks, while its support for CNNs enhances image classification tasks. Its scalable architecture allows for training on diverse hardware configurations, essential for large datasets. Additionally, TensorFlow integrates well with tools like TensorBoard and TensorFlow Lite for visualization, monitoring, and deployment on mobile devices. Its large, active community provides extensive resources, accelerating research and development. This thesis utilizes TensorFlow to develop a stacked model architecture for waste classification, aiming to improve training efficiency and accuracy through advanced features and capabilities, ultimately creating a robust and scalable system.

\subsection{Keras}
Keras [8], a high-level neural networks API written in Python, is designed for fast experimentation and prototyping, making it ideal for deep learning projects like waste classification. Its user-friendly, modular, and extensible nature simplifies building and testing complex neural network architectures, even for those with limited programming experience. Keras's integration with TensorFlow leverages the latter’s high performance and scalability, facilitating efficient handling of large datasets and complex computations. It also offers pre-trained models that are useful for transfer learning, saving time and computational resources. Supported by extensive documentation and a large community, Keras provides invaluable resources for development and troubleshooting. In this thesis, Keras is used to develop a stacked model architecture for waste classification, improving training efficiency and accuracy by utilizing its advanced features and capabilities.

\subsection{Scikit Learn}
Scikit-learn [9], a powerful open-source machine learning library in Python, is widely used for its simplicity and efficiency in data analysis and modeling. It provides a range of supervised and unsupervised learning algorithms through a consistent interface, making it accessible for both beginners and experts in machine learning. Scikit-learn excels in tasks such as classification, regression, clustering, and dimensionality reduction, and integrates well with other scientific Python libraries like NumPy and pandas. In waste classification, scikit-learn can be utilized for preprocessing data, training models, and evaluating their performance. Its robust toolkit allows for the implementation of various algorithms, enabling researchers to experiment and identify the most effective methods for accurate and efficient waste sorting. This versatility and ease of use make scikit-learn a valuable asset in developing scalable and efficient machine learning solutions for waste management applications.

\subsection{Numpy}
NumPy [10], a fundamental library for numerical computing in Python, provides support for arrays, matrices, and a collection of mathematical functions to operate on these data structures efficiently. Fundamental to the scientific Python ecosystem, NumPy enables high-performance operations on huge datasets, which makes it necessary for machine learning and data analysis jobs. While its interaction with other libraries like pandas, scikit-learn, and TensorFlow makes it an essential tool for building machine learning models, its array operations provide quick and flexible data handling. NumPy is used to prepare and convert unprocessed data into a format that is appropriate for model training in the context of waste classification. Large dataset handling done well by it guarantees fast completion of complicated calculations, which promotes the creation of reliable and scalable classification systems. Researchers can ensure precise and effective processing.

\subsection{Matplotlib}
Matplotlib [11] is a feature-rich Python charting package that is essential for scientific computing and machine learning data visualization. Researchers may efficiently depict data trends, distributions, and linkages with its adaptable platform that allows for the creation of static, animated, and interactive graphs. Matplotlib's diverse range of charting capabilities, including line, scatter, histogram, and heatmap plots, make it an indispensable tool for analyzing data and presenting conclusions. Matplotlib can be used in the field of waste classification to show the outcomes of data pretreatment procedures, track the performance of classification models over time, and depict the distribution of various trash categories. Matplotlib facilitates more informed decision-making and increases overall impact by helping academics and stakeholders comprehend and interpret complicated datasets through the use of graphic presentations of data and results.

\subsection{OpenCV}
OpenCV [12] (Open Source Computer Vision Library) is a highly versatile and widely used open-source library for computer vision and image processing tasks. It provides an extensive suite of tools and functions for image and video analysis, including object detection, motion tracking, image segmentation, and feature extraction. OpenCV is optimized for real-time applications, making it suitable for implementing fast and efficient image processing pipelines. In waste classification, OpenCV can be utilized to preprocess images, detect and segment waste objects, and extract relevant features for further analysis. Its ability to handle complex image transformations and processing tasks enhances the accuracy and efficiency of waste sorting systems. By leveraging OpenCV's capabilities, researchers can develop robust computer vision models that automate the identification and classification of waste materials, contributing to more effective and scalable waste management solutions.

\subsection{Colab}
Google Colaboratory (Colab) [13] is a free, cloud-based platform that provides users with access to powerful computational resources, including GPUs and TPUs, to facilitate machine learning and data analysis tasks. Colab allows users to write and execute Python code directly in their web browser, making it an accessible and convenient tool for researchers, educators, and practitioners. It supports various deep learning frameworks, such as TensorFlow, Keras, and PyTorch, enabling the development, training, and testing of complex neural network models. One of the key advantages of Colab is its ability to leverage high-performance hardware, like the Nvidia T4 GPU, which significantly speeds up the training process for large datasets and sophisticated models. Additionally, Colab's integration with Google Drive allows for seamless data storage and collaboration, making it an ideal environment for both individual projects and collaborative research efforts. Overall, Colab democratizes access to advanced computational tools, supporting the rapid advancement of machine learning and AI research.

\subsection{T4 GPU}
Google Colaboratory (Colab) provides access to the Nvidia T4 GPU [14], which significantly accelerates the training and execution of deep learning models. In the context of deep learning applications, the T4 GPU offers substantial performance improvements over traditional CPUs. The T4 GPU demonstrated a 25x speedup over the CPU. This enhancement is particularly beneficial for image classification tasks, where large datasets and complex neural network architectures require extensive computational resources. Colab's integration with powerful GPUs like the T4 facilitates faster model convergence and higher accuracy, making it an invaluable tool for researchers and practitioners in the field of machine learning and AI. In our study, we leverage the T4 GPU on Colab to train our waste classification models, enabling efficient handling of large waste datasets and significantly improving the training speed and model performance.


\section{Chapter Summary}
provides a thorough examination of waste management issues, especially in urban areas like Dhaka, Bangladesh, where inadequate infrastructure makes it difficult to handle enormous amounts of waste. In this chapter, we will look at the potential of computer vision technology to automate waste classification operations and how inefficient waste management techniques impact the environment, public health, and the economy.  addresses the need for more sustainable waste management systems and analyzes important problems such as data management, waste diversity, and computing limits. It also outlines the study aims and expected achievements. Moreover, the field of research, specifically emphasizing artificial intelligence and computer vision technologies. Highlights the need of utilizing tools such as TensorFlow, Keras, scikit-learn, NumPy, Matplotlib, OpenCV, Google Colab, and T4 GPU to tackle these difficulties.
